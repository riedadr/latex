\documentclass{article}
\usepackage[ngerman]{babel}
\usepackage[a4paper, left=3cm, right=3cm, top=2.5cm, headheight=110pt]{geometry}

\usepackage{cmbright}
\usepackage[OT1]{fontenc}

\usepackage{xcolor}
\usepackage{amsmath}
\usepackage[hidelinks]{hyperref}
\usepackage{siunitx}

\usepackage{fancyhdr}
\pagestyle{fancy}
\fancyhead[L]{Einführung in \LaTeX \\ Vorlesung 4}
\fancyhead[C]{{Mathematiksatz}\\}
\fancyhead[R]{Adrian Riedel \\ 8. März 2022}

%zwei Möglichkeiten, um Variablen zu erstellen
\newcommand{\pbsb}{pneumatischbetriebene Schaumstoffbandsäge}
\newcommand\rot[1]{\textcolor{red}{#1}}



\begin{document}
Diese \pbsb{} wurde über eine Variable (bzw. Command) eingefügt.
Die \pbsb{} ist \rot{rot}

\part*{Mathe}

Mithilfe des Satz des Pythagoras kann mit Hilfe der Kantenlängen eines Dreiecks die Hypothenus berechnen:
$a^2 + b^2 = c^2$.
\section*{Gaußsche Summenformel}
Durch das Summenzeichen kann man ausdrücken, dass Zahlen addiert werden soll, ohne diese Zahlen einzeln aufschreiben zu müssen.
\( \sum^n_1 i \).

\begin{displaymath}
    n = \sum^n_{i=0} i = \frac{n(n+1)}{2}
\end{displaymath}
Hier wird zum Beispiel ein Weg dargestelle, wie man ganz schnell alle natürlichen Zahlen bis zu einer bestimmten Zahl addiert.


\begin{align}
    a     & = b, & b = c \\
    b = c & = a
\end{align}

\section*{Symbole}
Mit Hilfe von \href{https://detexify.kirelabs.org/classify.html}{Detextify}
kann man einfache Symbole zeichnen und bekommt dann die Bezeichnung und das Paket des Symbols.

\begin{displaymath}
    \vec a =
    \int^{a}_{b}
    { 3 * \left(
        \frac{\prod^{n}_{i=1}{n}}{\sqrt[3]{8}}
    \right) }
\end{displaymath}

\[ m = \left(
    \begin{matrix}

            a_1 & a_2 & a_3 \\
            b_1 & b_2 & b_3 \\
            c_1 & c_2 & c_3
        \end{matrix}
    \right)
\]

\section*{Einheiten}
Die Fallbeschleunigung in Europa beträgt im Schnitt
\SI[per-mode=symbol]{9.81}{\metre\per\sec^{2}}

\end{document}