\documentclass{article}

\usepackage[a4paper, left=3cm, right=3cm, top=2.5cm, headheight=110pt]{geometry}
\usepackage[ngerman]{babel}

\usepackage{enumitem}
\usepackage[hidelinks]{hyperref}

\usepackage[math]{kurier}
\usepackage[T1]{fontenc}

\usepackage{blindtext}

%\usepackage{polyglossia} nur LuaTex
\usepackage{csquotes}

\usepackage{fancyhdr}
\pagestyle{fancy}
\fancyhead[L]{Einführung in \LaTeX \\ Übung 3}
\fancyhead[R]{Adrian Riedel \\ 8. März 2022}

\begin{document}
\section*{Übung 3.1}

\begin{enumerate}[label=\alph*)]
    \item \underline{
              \textbf{\textsc{Blindtext}}
              (\href{https://tug.org/FontCatalogue/kurier/}{\textit{Schriftart: Kurier}})
          }

          \blindtext[2]
\end{enumerate}

\section*{Übung 3.2:  \emph{\normalsize{Allgemeine Formatierung}}}
In \LaTeX gibt es unter Anderem folgende Formatierungsmöglichkeiten:
\begin{itemize}
    \item Listen
          \begin{description}
              \item [\small{- Ungeordnet}] \tiny{mit voranstehendem Symbol, z. B. \enquote{-}, \enquote{•} oder \enquote{$\ast$}}
              \item [\small{- Geordnet}] \tiny{durchnummeriert mit Zahlen oder Buchstaben}
          \end{description}
    \item Absätze
    \item Hervorhebungen
          \begin{itemize}
              \item \textbf{fettgedruckt}
              \item \textit{kursiv}
              \item \underline{unterstrichen}
          \end{itemize}
\end{itemize}

\end{document}