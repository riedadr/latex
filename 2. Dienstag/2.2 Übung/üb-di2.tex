\documentclass{article}

%? Pakete für Sprache und Layout
\usepackage[ngerman]{babel}
\usepackage[a4paper, left=3cm, right=3cm, top=2.5cm, headheight=110pt]{geometry}

%? Pakete für Aufgaben
\usepackage{xcolor}
\usepackage{amsmath}
\usepackage{siunitx}    %[binary-units] muss nicht mehr angegeben werden

%? Pakete für Optik
\usepackage{cmbright}
\usepackage[OT1]{fontenc}
\usepackage{enumitem}
\usepackage{fancyhdr}

% Kopfzeile
\pagestyle{fancy}
\fancyhead[L]{Einführung in \LaTeX \\ Übung 4}
\fancyhead[C]{{Mathematiksatz}\\}
\fancyhead[R]{Adrian Riedel \\ 8. März 2022}

%! für Übung 4.1 a)
\newcommand\pbsb[1]{\textcolor{#1}{pneumatischbetriebene Schaumstoffbandsäge}}
\newcommand{\empha}{empasize}


\begin{document}
\section*{Übung 4.1: Eigene Befehle}
\begin{enumerate}[label=\alph*)]
    \item \pbsb{red}
    \item Der Befehl \emph{emph} ist bereits vergeben.
          Man müsste den eigenen Befehl anders benennen, z. B. \emph{empha}.
          Der bestehende Befehl kann aber auch mit \texttt{\textbackslash{}renewcommand\{\textbackslash{}emph\}\{empasize\}} überschreiben
\end{enumerate}

\section*{Übung 4.2: Formeln}
\begin{enumerate}[label=\alph*)]
    \item Maxwell-Gleichungen
          \begin{align}
              \nabla \times \vec{E} & = - \frac{\partial \vec{B}}{\partial t}         &
              \nabla \times \vec{B} & = \vec{j} + \frac{\partial \vec{E}}{\partial t}   \\
              \nabla \cdot \vec{E}  & = \rho                                          &
              \nabla \cdot \vec{B}  & = 0
          \end{align}

    \item Shannon Entropy
          \begin{displaymath}
              H(X) = - \sum^{n}_{i=1}{P}(x_i) \log P(x_i)
          \end{displaymath}

    \item Skalarmultiplikation
          \begin{displaymath}
              \alpha \cdot A =
              \alpha \cdot \left(
              \begin{matrix}
                  a_{11} & \hdots & a_{1n} \\
                  \vdots & \ddots & \vdots \\
                  a_{m1} & \hdots & a_{mn}
              \end{matrix}
              \right) =
              \left(
              \begin{matrix}
                  \alpha \cdot a_{11} & \hdots & a_{1n} \\
                  \vdots              & \ddots & \vdots \\
                  \alpha \cdot a_{m1} & \hdots & a_{mn}
              \end{matrix}
              \right)
          \end{displaymath}
\end{enumerate}

\section*{Übung 4.3: Inline-Formeln}
Wir sollen die Formel \( \frac{1}{2} \cdot \sum^{N}_{n=0} g_{n}(x) = \int^{b}_{a} f(x) dx\) nehmen und in einen Fließtext schreiben.

\section*{Übung 4.4: Einheiten}
\begin{enumerate}[label=\alph*)]
    \item Frequenzen \\
          Das HackRF arbeitet von 0.1 \si{\giga\hertz} bis 6 \si{\giga\hertz}

    \item Binär \\
          $x = \SI{1}{\giga\byte}$ oder $ \SI{1}{\gibi\byte}$ \small({Die Binäreinheiten müssen bei \emph{siunitx} nicht mehr angegeben werden})

    \item Sonstige
          \begin{itemize}
              \item \SI{10}{\gram}
              \item \SI{16.7}{\metre\per\s}
              \item \SI{90}{\degree}
              \item \SI{100}{\percent}
              \item \SI{0.13}{\metre}, \SI{0.67}{\metre} und \SI{0.80}{\metre}
          \end{itemize}
\end{enumerate}

\end{document}