\documentclass[xetex]{beamer}
\usepackage{fontspec, listings, color}

\usetheme{metropolis}

\setmonofont[Scale=0.8]{Cascadia Code}

\lstset{
    keywordstyle=\color{blue},
    language=Java,
}

\title{Übung 9: Präsentation in \LaTeX}

\begin{document}
\maketitle
\begin{frame}{Inhalt}
    \tableofcontents
\end{frame}

\section{Einleitung}
\begin{frame}{Was ist das?}
    Moin erstmal. \\
    Diese Präsentation dient zum Kennenlernen der Möglichkeiten von Präsentationen in \LaTeX.
\end{frame}

\begin{frame}{Was macht das?}
    \structure{Wir schauen uns erstmal ein paar Grundlagen an:}
    \begin{itemize}
        \item Dokumentenklasse \emph{beamer}
        \item Verwenden von Themes
        \item Erstellen einer Titelseite sowie Inhaltsverzeichnis
    \end{itemize}
\end{frame}

\section{Ein bissle Code}
\begin{frame}[fragile]{Codebeispiel - lstlisting}
    Das ist zwar nicht die dritte Seite, aber hier kommt Code:

    \begin{lstlisting}

    public static void hello {
            System.out.println("Hello World!");
        }
    \end{lstlisting}
\end{frame}






\end{document}