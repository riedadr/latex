% !TEX program = xelatex
% !TEX encoding = UTF-8 Unicode
% !TEX spellcheck = de_DE

\documentclass{scrreprt}
\usepackage[ngerman]{babel}


\usepackage{blindtext,booktabs,color,nicefrac,polyglossia,setspace,xltxtra,yfonts,hyperref}
\usepackage{listings} %code highlighter
\usepackage{color} %use color
\usepackage{acronym}


\setmainlanguage{german}

\setromanfont[Mapping=tex-text]{Linux Libertine O}
\setsansfont[Mapping=tex-text]{Linux Biolinum O}

\onehalfspacing




\begin{document}

\subsection*{Abstract}
Dieses Dokument dient der Übung des Satzes von umfangreichen Projekten in \LaTeX\index{LaTeX@\LaTeX}. Es gehört zum siebten Übungszettel des \LaTeX-Kurses im Wintersemester 2016\,/\,17. Inhaltlich hat es so ziemlich nichts zu bieten, es könnte aber interessant sein, sich den zugehörigen Sourcecode mal anzusehen, da er eine Menge interessanter \LaTeX-Kommandos enthält.



\titlehead{\Large Universität Schlauenheim}
\subject{Masterarbeit}
\title{Risikomanagement in Zeiten von Social Media}
\subtitle{Design interaktiver Apps für Banken und
Versicherungen}
\author{cand.\,stup. Uli Ungenau}
\date{30. Februar 2017}
\publishers{Betreut durch Prof.\,Dr.\,rer.\,stup. Naseweis}
\dedication{Für meine Mama.}
\maketitle

\setchapterpreamble[o]{\dictum[W. Busch]{Stets findet Überraschung statt. Da, wo man's nicht erwartet hat.}}

\chapter{Einleitung}

\blindtext$\sin(x)\cdot\cos(x) = -\nicefrac{1}{2} \cos(2x)$\footnote{Man baechte auch, dass $\sin(x\pm y) = \sin(x)\cos(y) \pm \cos(x)\sin(y)$}

\begin{table}[h]
  \centering
  \begin{tabular}{ccc}
    \toprule
    a & b & c\\
    d & e & f\\
    g & h & i\\
    \bottomrule
  \end{tabular}
  \caption{Die erste Tabelle}
  \label{tabelle1}
\end{table}
%Define the listing package

\definecolor{mygreen}{rgb}{0,0.6,0}
\definecolor{mygray}{rgb}{0.5,0.5,0.5}
\definecolor{mymauve}{rgb}{0.58,0,0.82}

%Customize a bit the look
\lstset{ %
    backgroundcolor=\color{white}, % choose the background color; you must add \usepackage{color} or \usepackage{xcolor}
    basicstyle=\footnotesize, % the size of the fonts that are used for the code
    breakatwhitespace=false, % sets if automatic breaks should only happen at whitespace
    breaklines=true, % sets automatic line breaking
    captionpos=b, % sets the caption-position to bottom
    commentstyle=\color{mygreen}, % comment style
    deletekeywords={...}, % if you want to delete keywords from the given language
    escapeinside={\%*}{*)}, % if you want to add LaTeX within your code
    extendedchars=true, % lets you use non-ASCII characters; for 8-bits encodings only, does not work with UTF-8
    frame=single, % adds a frame around the code
    keepspaces=true, % keeps spaces in text, useful for keeping indentation of code (possibly needs columns=flexible)
    keywordstyle=\color{blue}, % keyword style
    % language=Octave, % the language of the code
    morekeywords={*,...}, % if you want to add more keywords to the set
    numbers=left, % where to put the line-numbers; possible values are (none, left, right)
    numbersep=5pt, % how far the line-numbers are from the code
    numberstyle=\tiny\color{mygray}, % the style that is used for the line-numbers
    rulecolor=\color{black}, % if not set, the frame-color may be changed on line-breaks within not-black text (e.g. comments (green here))
    showspaces=false, % show spaces everywhere adding particular underscores; it overrides 'showstringspaces'
    showstringspaces=false, % underline spaces within strings only
    showtabs=false, % show tabs within strings adding particular underscores
    stepnumber=1, % the step between two line-numbers. If it's 1, each line will be numbered
    stringstyle=\color{mygreen}, % string literal style
    tabsize=2, % sets default tabsize to 2 spaces
    title=\lstname % show the filename of files included with \lstinputlisting; also try caption instead of title
}
%END of listing package%

\definecolor{darkgray}{rgb}{.4,.4,.4}
\definecolor{purple}{rgb}{0.65, 0.12, 0.82}

\lstdefinelanguage{JavaScript}{
  morekeywords=[1]{break, continue, delete, else, for, function, if, in,
    new, return, this, typeof, var, void, while, with},
  % Literals, primitive types, and reference types.
  morekeywords=[2]{false, null, true, boolean, number, undefined,
    Array, Boolean, Date, Math, Number, String, Object},
  % Built-ins.
  morekeywords=[3]{eval, parseInt, parseFloat, escape, unescape},
  sensitive,
  morecomment=[s]{/*}{*/},
  morecomment=[l]//,
  morecomment=[s]{/**}{*/}, % JavaDoc style comments
  morestring=[b]',
  morestring=[b]"
}[keywords, comments, strings]

\chapter{Codeschnipsel}
\section{Beispiel in ReactJS}

\lstinputlisting[language=JavaScript, firstline=9]{content/code/Aktuell.jsx}
\setchapterpreamble[o]{\dictum[Cato]{\textsc{Nullus est liber tam malus, ut non aliqua parte prosit.}}}

\index{Blinddokument|(}
\blinddocument
\index{Blinddokument|)}

\begin{figure}[h]
  \centering
  \fbox{I am a picture!}
  \caption{Ein Bild, das die Aussage des Textes unterstreicht.}
  \label{statement}
\end{figure}






\setchapterpreamble[o]{\dictum[U.\,R. Heber]{Ein schlauer Spruch bereichtert den Kapitelanfang.}}

\chapter{Ein weiteres Kapitel}

\fbox{I am a picture, introducing this chapter!}

\label{introduction}

\index{Blindtext}
\Blindtext\footnote{Man beachte auch, dass $\sin(x\pm y) = \sin(x)\cos(y) \pm \cos(x)\sin(y)$}

\begin{table}[h]
  \centering
  \begin{tabular}{ccc}
  \toprule
  eins & zwei & drei\\
  vier & fünf & sechs\\
  sieben & acht & neun\\
  \bottomrule
  \end{tabular}
  \caption{Eine Tabelle mit neun Einträgen}
  \label{tabelle3}
\end{table}

\setchapterpreamble[o]{\dictum[F. Halm]{\hspace*{2em}\textfrak{Ruhe bleibt den Leichen;\\ Der Lebende tauch' frisch ins: Lebens:meer.}}}


\chapter{Und noch ein weiteres Kapitel}

\begin{figure}[p]
  \centering
  \fbox{I am a picture!}
  \caption{Beispiel zu diesem Kapitel}
  \label{example}
\end{figure}

\index{Blindtext}
\Blindtext Der \verb$\Bindtext$-Befehl\index{Blindtext} ist eine nette Sache, wenn man in \LaTeX\index{LaTeX@\LaTeX} sehen will, wie ein Dokument mit viel Inhalt aussieht, ohne, dass man Inhalt hat.

\begin{figure}[p]
  \centering
  \fbox{I am a picture!}
  \caption{Veranschaulichung der Aussage}
  \label{illustration}
\end{figure}

\begin{figure}[p]
  \centering
  \fbox{I am a picture!}
  \caption{Detailansicht}
  \label{detail}
\end{figure}

\begin{figure}[p]
  \centering
  \fbox{I am a picture!}
  \caption{Visualisierung des Ergebnisses}
  \label{visualization}
\end{figure}
\chapter{Abkürzungen}
Dieser Text enthält Abkürzungen, wie beispielsweise \ac{TCP} oder \ac{EU}. Aber auch die \ac{PBSB} steht hier.
Wenn ich die \ac{PBSB} dann nochmal erwähne, wird die volle Bezeichnung nicht mehr angegeben.
\begin{acronym}[TCP] %% Erstellt auch automatisch Abkürzungsverzeichnis
    \acro{TCP}{Transmission Controll Protocol}
    \acro{EU}{European Union}
    \acro{PBSB}{Pneumatischbetriebene Schaumstoffbandsäge}
\end{acronym}



\end{document}
