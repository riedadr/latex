\documentclass{article}

%? Pakete für Sprache und Layout
\usepackage[ngerman]{babel}
\usepackage[a4paper, left=3cm, right=3cm, top=2.5cm, headheight=110pt]{geometry}

%? Pakete für Aufgaben
\usepackage{tikz}
\usetikzlibrary{trees}
\usepackage{pgfplots}
\pgfplotsset{compat=newest}

%? Pakete für Optik
\usepackage{cmbright}
\usepackage[OT1]{fontenc}
\usepackage{enumitem}
\usepackage{fancyhdr}
\usepackage{csquotes}


% Kopfzeile
\pagestyle{fancy}
\fancyhead[L]{Einführung in \LaTeX \\ Übung 6}
\fancyhead[C]{{Gleitumgebung}\\}
\fancyhead[R]{Adrian Riedel \\ 9. März 2022}


\begin{document}
\section*{Übung 6.1: Baum mit TikZ}
\begin{enumerate}[label=\alph*)]
    \item Biere \\
          \begin{figure}[h]
              \centering
              \begin{tikzpicture}[
                      edge from parent fork down,
                      level 1/.style={sibling distance=5cm},
                      level 2/.style={sibling distance=2cm},
                  ]
                  \node {Bier}
                  child {node {obergärig}
                          child {node {Alt}}
                          child {node {Kölsch}}
                          child {node {Weizen}}
                      }
                  child {node {untergärig}
                          child {node {Märzen}}
                          child {node {Pils}}
                      }
                  ;
              \end{tikzpicture}
              \caption{Stammbaum einiger Bierstile}
          \end{figure}
\end{enumerate}

\section*{Übung 6.2: Daten darstellen mit pgfplots}
\begin{enumerate}[label=\alph*)]
    \item Picture 1 \\

              \begin{figure}[h]
                \centering
                  \begin{tikzpicture}[scale=.5]
                      \begin{axis}[
                              width=\linewidth, % Scale the plot to \linewidth
                              grid=major, % Display a grid
                              grid style={line width=.1pt, draw=gray!30}, % Set the style
                              xlabel= $x$, % Set the labels
                              xtick={1,2,3,4,5},
                              ylabel= $y$,
                              ytick={2,4,6,8},
                          ]
                          \addplot
                          table[x=x,y=y1,col sep=comma] {data.csv};
                          \addlegendentry{$y1$}
                          \addplot
                          table[x=x,y=y2,col sep=comma] {data.csv};
                          \addlegendentry{$y2$}
                          \addplot
                          table[x=x,y=y3,col sep=comma] {data.csv};
                          \addlegendentry{$y2$}
                      \end{axis}
                  \end{tikzpicture}
                  \caption{Messergebnisse}
              \end{figure}
\end{enumerate}




\end{document}